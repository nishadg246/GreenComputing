% You should title the file with a .tex extension (hw1.tex, for example)
\documentclass[11pt]{article}

\usepackage{amsmath}
\usepackage{amssymb}
\usepackage{fancyhdr}

\oddsidemargin0cm
\topmargin-2cm     %I recommend adding these three lines to increase the 
\textwidth16.5cm   %amount of usable space on the page (and save trees)
\textheight23.5cm  

\newcommand{\question}[2] {\vspace{.25in} \hrule\vspace{0.5em}
	\noindent{\bf #1: #2} \vspace{0.5em}
	\hrule \vspace{.10in}}
\renewcommand{\part}[1] {\vspace{.10in} {\bf (#1)}}

\newcommand{\myname}{Nishad Gothoskar}
\newcommand{\myandrew}{ngothosk}
\newcommand{\myhwnum}{2}

\setlength{\parindent}{0pt}
\setlength{\parskip}{5pt plus 1pt}

\pagestyle{fancyplain}
\lhead{\fancyplain{}{\textbf{HW\myhwnum}}}      % Note the different brackets!
\rhead{\fancyplain{}{\myname\\ \myandrew}}
\chead{\fancyplain{}{Green Computing}}

\begin{document}
	
	\medskip                        % Skip a "medium" amount of space
	% (latex determines what medium is)
	% Also try: \bigskip, \littleskip
	
	\thispagestyle{plain}
	\begin{center}                  % Center the following lines
		{\Large DoubleDip: Leveraging Thermoelectric Harvesting for Low Power Monitoring of Sporadic Water Use
} \\
		\myname \\
		\myandrew @andrew.cmu.edu\\
	\end{center}
	
	\question{1}{Summary}
	\quad This a heavily hardware based project. I really enjoyed it, especially because of this sentence: "the key idea of the DoubleDip design is that it relies on an artifact of the phenomenon it is trying to capture to wake the system out of deep sleep through thermoelectric harvesting, removing the need for duty cycling, saving state, and preventing cold-boot delays"
	
	\quad The sensor system is composed of two main parts. The first is the low-power wake up system. Rather than have the entire system on at all times, the wake up system waits for the stimulus to wake up the sensor system. They are looking for changes in water flow, so they have a TEG that can sense a voltage change caused by a temperature gradient in the water flow. It waits for this to wake up the entire system which involves sensors of vibration.
	
	\quad The most interesting  part of this project is that it is self-powered. They harvest energy that can be gathered from these temperature gradients to generate power for the devices. It can also collect and store energy for later use. In addition they have a buddy charging system that allows other sensors (in close proximity, so usually hot and cold pipes that are next to each other) to share and recharge and supply energy to one an other through a request system.
	
	\quad The novel ideas they present are thermoelectric wakeup, harvesting, and comprehensive testing of this system. They use batteries not capacitors. And the entire system is following a strict state machine with changes only causes by features detected by the sensors.
	
	\quad This system has no way of itself improving efficiency. Its current state is a measurement and instrumentation system. However, with it there is much room for energy efficiency. Because being able to see where water is being used and when it is being used gives users more insight and feedback from their usage.
	\question{2}{Strengths}
	\begin{itemize}
		\item Self sufficient
		\item Insight into something sometimes ignored (water) when it comes to the "green" movement
	\end{itemize}
	\question{3}{Weaknesses}
	\begin{itemize}
		\item No user friendly UI / feedback to make them change
	\end{itemize}
	\question{4}{Future Directions}
	\begin{itemize}
		\item Apply this sensing technique to other realms
	\end{itemize}

\end{document}


