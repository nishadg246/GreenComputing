
\documentclass[journal]{IEEEtran}
\usepackage{blindtext}
\usepackage{graphicx}


\hyphenation{op-tical net-works semi-conduc-tor}
\begin{document}
%
% paper title
% can use linebreaks \\ within to get better formatting as desired
\title{EnergyAware: Personal Energy Monitoring System }

\author{\textbf{Nishad Gothoskar}\\Carnegie Mellon University\\ngothosk@andrew.cmu.edu}
\markboth{Journal of \LaTeX\ Class Files,~Vol.~6, No.~1, January~2007}%
{Shell \MakeLowercase{\textit{et al.}}: Bare Demo of IEEEtran.cls for Journals}
\maketitle


\begin{abstract}

\end{abstract}


% Note that keywords are not normally used for peerreview papers.
\begin{IEEEkeywords}
IEEEtran, journal, \LaTeX, paper, template.
\end{IEEEkeywords}


\IEEEpeerreviewmaketitle



\section{Introduction}
Many times in the fields of Smart Buildings and Smart Homes, we disregard the essential concept of Energy Proportionality. In commercial buildings and residential homes, it may be true that there is a high inherent overhead and therefore proportionality cannot be achieved. But on a smaller scale, like apartments and single bedrooms that most young people live in, this proportionality is truly achievable, given that there is much less energy overhead. When we are not at home, our energy usage should be close to nothing and it should increase when we are present. But, the fact that our energy usage is not proportional should not be blamed on the users themselves. In fact many users would be readily willing to reduce their waste usage and adjust their overall usage. But power and energy usage measurement and statistics are not readily available or accessible to users, so they have no way of knowing how or in what ways they should change. In this paper we look to change this so users can be more aware of their energy consumption.

\subsection{Jevon's Paradox}
Jevon’s paradox says that when the efficiency of usage of a resource is increased there will be an increase in usage and demand and in effect there will be little to no change. While this may not be completely true when it comes to energy resources, it may be limiting our potential of energy savings. There is an aspect to Green Computing that can’t be solved by technological innovations alone. In some ways, significant improvement is dependent on changing user's usage and  their demand habits. 

\subsection{Belkin Wemo}
Our measurement sensors are Belkin Wemo Insight Switches. These switches come with free software through an iOS and Android application. The application provides the ability to see minimal statistics about each plug and also control it remotely. It is a quite thorough and useful application

\section{Conclusion}
\blindtext





% if have a single appendix:
%\appendix[Proof of the Zonklar Equations]
% or
%\appendix  % for no appendix heading
% do not use \section anymore after \appendix, only \section*
% is possibly needed

% use appendices with more than one appendix
% then use \section to start each appendix
% you must declare a \section before using any
% \subsection or using \label (\appendices by itself
% starts a section numbered zero.)
%


\appendices
\section{Proof of the First Zonklar Equation}
Some text for the appendix.

% use section* for acknowledgement
\section*{Acknowledgment}


The authors would like to thank...


% Can use something like this to put references on a page
% by themselves when using endfloat and the captionsoff option.
\ifCLASSOPTIONcaptionsoff
  \newpage
\fi



% trigger a \newpage just before the given reference
% number - used to balance the columns on the last page
% adjust value as needed - may need to be readjusted if
% the document is modified later
%\IEEEtriggeratref{8}
% The "triggered" command can be changed if desired:
%\IEEEtriggercmd{\enlargethispage{-5in}}

% references section

% can use a bibliography generated by BibTeX as a .bbl file
% BibTeX documentation can be easily obtained at:
% http://www.ctan.org/tex-archive/biblio/bibtex/contrib/doc/
% The IEEEtran BibTeX style support page is at:
% http://www.michaelshell.org/tex/ieeetran/bibtex/
%\bibliographystyle{IEEEtran}
% argument is your BibTeX string definitions and bibliography database(s)
%\bibliography{IEEEabrv,../bib/paper}
%
% <OR> manually copy in the resultant .bbl file
% set second argument of \begin to the number of references
% (used to reserve space for the reference number labels box)
\begin{thebibliography}{1}

\bibitem{IEEEhowto:kopka}
H.~Kopka and P.~W. Daly, \emph{A Guide to \LaTeX}, 3rd~ed.\hskip 1em plus
  0.5em minus 0.4em\relax Harlow, England: Addison-Wesley, 1999.

\end{thebibliography}

% biography section
% 
% If you have an EPS/PDF photo (graphicx package needed) extra braces are
% needed around the contents of the optional argument to biography to prevent
% the LaTeX parser from getting confused when it sees the complicated
% \includegraphics command within an optional argument. (You could create
% your own custom macro containing the \includegraphics command to make things
% simpler here.)
%\begin{biography}[{\includegraphics[width=1in,height=1.25in,clip,keepaspectratio]{mshell}}]{Michael Shell}
% or if you just want to reserve a space for a photo:


% You can push biographies down or up by placing
% a \vfill before or after them. The appropriate
% use of \vfill depends on what kind of text is
% on the last page and whether or not the columns
% are being equalized.

%\vfill

% Can be used to pull up biographies so that the bottom of the last one
% is flush with the other column.
%\enlargethispage{-5in}



% that's all folks
\end{document}


